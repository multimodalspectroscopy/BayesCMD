%% Generated by Sphinx.
\def\sphinxdocclass{report}
\documentclass[letterpaper,10pt,english]{sphinxmanual}
\ifdefined\pdfpxdimen
   \let\sphinxpxdimen\pdfpxdimen\else\newdimen\sphinxpxdimen
\fi \sphinxpxdimen=.75bp\relax

\usepackage[utf8]{inputenc}
\ifdefined\DeclareUnicodeCharacter
 \ifdefined\DeclareUnicodeCharacterAsOptional
  \DeclareUnicodeCharacter{"00A0}{\nobreakspace}
  \DeclareUnicodeCharacter{"2500}{\sphinxunichar{2500}}
  \DeclareUnicodeCharacter{"2502}{\sphinxunichar{2502}}
  \DeclareUnicodeCharacter{"2514}{\sphinxunichar{2514}}
  \DeclareUnicodeCharacter{"251C}{\sphinxunichar{251C}}
  \DeclareUnicodeCharacter{"2572}{\textbackslash}
 \else
  \DeclareUnicodeCharacter{00A0}{\nobreakspace}
  \DeclareUnicodeCharacter{2500}{\sphinxunichar{2500}}
  \DeclareUnicodeCharacter{2502}{\sphinxunichar{2502}}
  \DeclareUnicodeCharacter{2514}{\sphinxunichar{2514}}
  \DeclareUnicodeCharacter{251C}{\sphinxunichar{251C}}
  \DeclareUnicodeCharacter{2572}{\textbackslash}
 \fi
\fi
\usepackage{cmap}
\usepackage[T1]{fontenc}
\usepackage{amsmath,amssymb,amstext}
\usepackage{babel}
\usepackage{times}
\usepackage[Bjarne]{fncychap}
\usepackage[dontkeepoldnames]{sphinx}

\usepackage{geometry}

% Include hyperref last.
\usepackage{hyperref}
% Fix anchor placement for figures with captions.
\usepackage{hypcap}% it must be loaded after hyperref.
% Set up styles of URL: it should be placed after hyperref.
\urlstyle{same}
\addto\captionsenglish{\renewcommand{\contentsname}{Contents:}}

\addto\captionsenglish{\renewcommand{\figurename}{Fig.}}
\addto\captionsenglish{\renewcommand{\tablename}{Table}}
\addto\captionsenglish{\renewcommand{\literalblockname}{Listing}}

\addto\captionsenglish{\renewcommand{\literalblockcontinuedname}{continued from previous page}}
\addto\captionsenglish{\renewcommand{\literalblockcontinuesname}{continues on next page}}

\addto\extrasenglish{\def\pageautorefname{page}}

\setcounter{tocdepth}{1}



\title{BayesCMD Documentation}
\date{Oct 06, 2017}
\release{}
\author{Joshua Russell-Buckland}
\newcommand{\sphinxlogo}{\vbox{}}
\renewcommand{\releasename}{Release}
\makeindex

\begin{document}

\maketitle
\sphinxtableofcontents
\phantomsection\label{\detokenize{index::doc}}


BayesCMD is a package intended to expand the capabilities of the
Brain/Circulation Modelling (BCMD) framework. It introduces the ability to
obtain posterior distributions for model parameters by using Approximate
Bayesian Computation (ABC).


\chapter{bcmdModel}
\label{\detokenize{bcmdModel:welcome-to-bayescmd-s-documentation}}\label{\detokenize{bcmdModel:bcmdmodel}}\label{\detokenize{bcmdModel::doc}}

\section{Running the BCMD Model}
\label{\detokenize{bcmdModel:module-bayescmd.bcmdModel.bcmd_model}}\label{\detokenize{bcmdModel:running-the-bcmd-model}}\index{bayescmd.bcmdModel.bcmd\_model (module)}\index{ModelBCMD (class in bayescmd.bcmdModel)}

\begin{fulllineitems}
\phantomsection\label{\detokenize{bcmdModel:bayescmd.bcmdModel.ModelBCMD}}\pysiglinewithargsret{\sphinxbfcode{class }\sphinxcode{bayescmd.bcmdModel.}\sphinxbfcode{ModelBCMD}}{\emph{model\_name}, \emph{inputs=None}, \emph{params=None}, \emph{times=None}, \emph{outputs=None}, \emph{burn\_in=999}, \emph{create\_input=True}, \emph{input\_file=None}, \emph{suppress=False}, \emph{workdir=None}, \emph{deleteWorkdir=False}, \emph{timeout=30}, \emph{basedir='../bayescmd'}, \emph{debug=False}, \emph{testing=False}}{}
BCMD model class. this can be used to create inputs, run simulations etc.
\index{create\_default\_input() (bayescmd.bcmdModel.ModelBCMD method)}

\begin{fulllineitems}
\phantomsection\label{\detokenize{bcmdModel:bayescmd.bcmdModel.ModelBCMD.create_default_input}}\pysiglinewithargsret{\sphinxbfcode{create\_default\_input}}{}{}
Method to create input file and write to string buffer for acces
direct from memory.

\end{fulllineitems}

\index{create\_initialised\_input() (bayescmd.bcmdModel.ModelBCMD method)}

\begin{fulllineitems}
\phantomsection\label{\detokenize{bcmdModel:bayescmd.bcmdModel.ModelBCMD.create_initialised_input}}\pysiglinewithargsret{\sphinxbfcode{create\_initialised\_input}}{}{}
Method to create input file and write to string buffer for access
direct from memory.

\end{fulllineitems}

\index{output\_parse() (bayescmd.bcmdModel.ModelBCMD method)}

\begin{fulllineitems}
\phantomsection\label{\detokenize{bcmdModel:bayescmd.bcmdModel.ModelBCMD.output_parse}}\pysiglinewithargsret{\sphinxbfcode{output\_parse}}{}{}
Function to parse the output files into a dictionary.

\end{fulllineitems}

\index{write\_default\_input() (bayescmd.bcmdModel.ModelBCMD method)}

\begin{fulllineitems}
\phantomsection\label{\detokenize{bcmdModel:bayescmd.bcmdModel.ModelBCMD.write_default_input}}\pysiglinewithargsret{\sphinxbfcode{write\_default\_input}}{}{}
Function to write a default input to file.

\end{fulllineitems}

\index{write\_initialised\_input() (bayescmd.bcmdModel.ModelBCMD method)}

\begin{fulllineitems}
\phantomsection\label{\detokenize{bcmdModel:bayescmd.bcmdModel.ModelBCMD.write_initialised_input}}\pysiglinewithargsret{\sphinxbfcode{write\_initialised\_input}}{}{}
Function to write a default input to file.

\end{fulllineitems}


\end{fulllineitems}



\section{Input Creation}
\label{\detokenize{bcmdModel:input-creation}}
Input files are required by the BCMD model.


\subsection{input\_creation}
\label{\detokenize{bcmdModel:module-bayescmd.bcmdModel.input_creation}}\label{\detokenize{bcmdModel:id1}}\index{bayescmd.bcmdModel.input\_creation (module)}
Create input files for use with a BCMD model.

Input files are needed in order to set model parameters and provide driving
inputs.
\index{InputCreator (class in bayescmd.bcmdModel)}

\begin{fulllineitems}
\phantomsection\label{\detokenize{bcmdModel:bayescmd.bcmdModel.InputCreator}}\pysiglinewithargsret{\sphinxbfcode{class }\sphinxcode{bayescmd.bcmdModel.}\sphinxbfcode{InputCreator}}{\emph{times}, \emph{inputs}, \emph{outputs=None}, \emph{params=None}, \emph{filename=None}}{}
Create an input file by passing relevant information to the class.

This input file is then used to create an input file that can either be
written to file or kept in buffer and passed directly to the model.
\begin{quote}\begin{description}
\item[{Parameters}] \leavevmode\begin{itemize}
\item {} 
\sphinxstyleliteralstrong{times} (\sphinxcode{list} of \sphinxcode{float} or \sphinxcode{int}) \textendash{} List of times at which measurement data has been collected and needs
to be simulated.

\item {} 
\sphinxstyleliteralstrong{inputs} (\sphinxstyleliteralemphasis{dict}) \textendash{} Dictionary of model inputs and their values. Has form
\{‘names’ : \sphinxcode{list} of \sphinxcode{str},
‘values’ : \sphinxcode{list} of \sphinxcode{list} of \sphinxcode{float}\}
where \sphinxtitleref{names} should be a list of each model input name, matching up to
the model inputs and \sphinxtitleref{values} would be a list of lsits, where each
sublist is the input values for that time point. With this in mind,
the length of \sphinxtitleref{inputs{[}‘values’{]}{}`} should equal length of \sphinxtitleref{times}.

\item {} 
\sphinxstyleliteralstrong{filename} (\sphinxcode{str}, optional) \textendash{} Name of the input file to be written to if writing to file is required.
Default is \sphinxcode{None}.

\item {} 
\sphinxstyleliteralstrong{params} (\sphinxcode{dict} of \sphinxcode{str}: \sphinxcode{float}, optional) \textendash{} Dictionary of \{‘parameter’: param\_value\}

\item {} 
\sphinxstyleliteralstrong{outputs} (\sphinxcode{list} of \sphinxcode{str}, optional) \textendash{} List of model outputs to return.

\end{itemize}

\end{description}\end{quote}
\index{times (bayescmd.bcmdModel.input\_creation.InputCreator attribute)}

\begin{fulllineitems}
\phantomsection\label{\detokenize{bcmdModel:bayescmd.bcmdModel.input_creation.InputCreator.times}}\pysigline{\sphinxbfcode{times}}
\sphinxcode{list} of \sphinxcode{float} or \sphinxcode{int} \textendash{} List of times at which measurement data has been collected and needs
to be simulated.

\end{fulllineitems}

\index{inputs (bayescmd.bcmdModel.input\_creation.InputCreator attribute)}

\begin{fulllineitems}
\phantomsection\label{\detokenize{bcmdModel:bayescmd.bcmdModel.input_creation.InputCreator.inputs}}\pysigline{\sphinxbfcode{inputs}}
\sphinxstyleemphasis{dict} \textendash{} Dictionary of model inputs and their values. Has form
\{‘names’ : \sphinxcode{list} of \sphinxcode{str},
‘values’ : \sphinxcode{list} of \sphinxcode{list} of \sphinxcode{float}\}
where \sphinxtitleref{names} should be a list of each model input name, matching up to
the model inputs and \sphinxtitleref{values} would be a list of lsits, where each
sublist is the input values for that time point. With this in mind,
the length of \sphinxtitleref{inputs{[}‘values’{]}{}`} should equal length of \sphinxtitleref{times}.

\end{fulllineitems}

\index{f\_out (bayescmd.bcmdModel.input\_creation.InputCreator attribute)}

\begin{fulllineitems}
\phantomsection\label{\detokenize{bcmdModel:bayescmd.bcmdModel.input_creation.InputCreator.f_out}}\pysigline{\sphinxbfcode{f\_out}}
\sphinxcode{StringIO()} \textendash{} String buffer object to which the input file will be written.

\end{fulllineitems}

\index{filename (bayescmd.bcmdModel.input\_creation.InputCreator attribute)}

\begin{fulllineitems}
\phantomsection\label{\detokenize{bcmdModel:bayescmd.bcmdModel.input_creation.InputCreator.filename}}\pysigline{\sphinxbfcode{filename}}
\sphinxstyleemphasis{str} \textendash{} Name of the input file to be written to if writing to file is required.
Default is \sphinxcode{None}.

\end{fulllineitems}

\index{params (bayescmd.bcmdModel.input\_creation.InputCreator attribute)}

\begin{fulllineitems}
\phantomsection\label{\detokenize{bcmdModel:bayescmd.bcmdModel.input_creation.InputCreator.params}}\pysigline{\sphinxbfcode{params}}
dict of \sphinxcode{str}: \sphinxcode{float}. \textendash{} Dictionary of \{‘parameter’: param\_value\}

\end{fulllineitems}

\index{outputs (bayescmd.bcmdModel.input\_creation.InputCreator attribute)}

\begin{fulllineitems}
\phantomsection\label{\detokenize{bcmdModel:bayescmd.bcmdModel.input_creation.InputCreator.outputs}}\pysigline{\sphinxbfcode{outputs}}
\sphinxcode{list} of \sphinxcode{str} \textendash{} List of model outputs to return.

\end{fulllineitems}

\index{default\_creation() (bayescmd.bcmdModel.InputCreator method)}

\begin{fulllineitems}
\phantomsection\label{\detokenize{bcmdModel:bayescmd.bcmdModel.InputCreator.default_creation}}\pysiglinewithargsret{\sphinxbfcode{default\_creation}}{}{}
Create a default input file from given arguments.

Assumes parameters remain unchanged from default values.
\begin{quote}\begin{description}
\item[{Returns}] \leavevmode
Returns the input file as a String.IO() buffer object.

\item[{Return type}] \leavevmode
\sphinxcode{String.IO()}

\end{description}\end{quote}

\end{fulllineitems}

\index{initialised\_creation() (bayescmd.bcmdModel.InputCreator method)}

\begin{fulllineitems}
\phantomsection\label{\detokenize{bcmdModel:bayescmd.bcmdModel.InputCreator.initialised_creation}}\pysiglinewithargsret{\sphinxbfcode{initialised\_creation}}{\emph{burn\_in}}{}
Create an input file from given arguments.

Creates an input file thatcan have non-default parameter values and
outputs, as well as a burn in period. Assumes parameters remain
constant for the full duration of the simulation.
\begin{quote}\begin{description}
\item[{Parameters}] \leavevmode
\sphinxstyleliteralstrong{burn\_in} (\sphinxcode{float} or \sphinxcode{int}) \textendash{} Length of burn in period at start of the simulation.

\item[{Returns}] \leavevmode
Returns the input file as a String.IO() buffer object.

\item[{Return type}] \leavevmode
\sphinxcode{String.IO()}

\end{description}\end{quote}

\end{fulllineitems}

\index{input\_file\_write() (bayescmd.bcmdModel.InputCreator method)}

\begin{fulllineitems}
\phantomsection\label{\detokenize{bcmdModel:bayescmd.bcmdModel.InputCreator.input_file_write}}\pysiglinewithargsret{\sphinxbfcode{input\_file\_write}}{}{}
Write input file from buffer to file.

\end{fulllineitems}


\end{fulllineitems}



\chapter{abc}
\label{\detokenize{abc:abc}}\label{\detokenize{abc::doc}}
The \sphinxtitleref{abc} subpackage is used to handle the Approximate Bayesian Computation
(ABC) specific components of BayesCMD. This includes running the model multiple
times in a batch process, calculating distances between datasets and generating
priors for parameters.
\phantomsection\label{\detokenize{abc:module-bayescmd.abc}}\index{bayescmd.abc (module)}

\section{Distances}
\label{\detokenize{abc:distances}}\label{\detokenize{abc:module-bayescmd.abc.distances}}\index{bayescmd.abc.distances (module)}
Use to generate distance measures between simulated and real time series.
\index{DISTANCES (in module bayescmd.abc.distances)}

\begin{fulllineitems}
\phantomsection\label{\detokenize{abc:bayescmd.abc.distances.DISTANCES}}\pysigline{\sphinxcode{bayescmd.abc.distances.}\sphinxbfcode{DISTANCES}}
\sphinxstyleemphasis{dict} \textendash{} Dictionary contianing the distance aliases, mapping to the functions.

\end{fulllineitems}

\index{Error}

\begin{fulllineitems}
\phantomsection\label{\detokenize{abc:bayescmd.abc.distances.Error}}\pysigline{\sphinxbfcode{exception }\sphinxcode{bayescmd.abc.distances.}\sphinxbfcode{Error}}
Base class for exceptions in this module.

\end{fulllineitems}

\index{ZeroArrayError}

\begin{fulllineitems}
\phantomsection\label{\detokenize{abc:bayescmd.abc.distances.ZeroArrayError}}\pysigline{\sphinxbfcode{exception }\sphinxcode{bayescmd.abc.distances.}\sphinxbfcode{ZeroArrayError}}
Exception raised for errors in the zero array.

\end{fulllineitems}

\index{check\_for\_key() (in module bayescmd.abc.distances)}

\begin{fulllineitems}
\phantomsection\label{\detokenize{abc:bayescmd.abc.distances.check_for_key}}\pysiglinewithargsret{\sphinxcode{bayescmd.abc.distances.}\sphinxbfcode{check\_for\_key}}{\emph{dictionary}, \emph{target}}{}
Check that a dictionary contains a key, and if so, return its data.
\begin{quote}\begin{description}
\item[{Parameters}] \leavevmode\begin{itemize}
\item {} 
\sphinxstyleliteralstrong{dictionary} (\sphinxstyleliteralemphasis{dict}) \textendash{} Dictionary to check for \sphinxtitleref{target} key.

\item {} 
\sphinxstyleliteralstrong{target} (\sphinxstyleliteralemphasis{str}) \textendash{} String containing the target variable that is expected to be found in
\sphinxtitleref{dictionary}

\end{itemize}

\item[{Returns}] \leavevmode
\sphinxstylestrong{data} \textendash{} List of data found in \sphinxtitleref{dictionary}. This is likely to be the time
series data collected experimentally or generated by the model.

\item[{Return type}] \leavevmode
list

\end{description}\end{quote}

\end{fulllineitems}

\index{euclidean\_dist() (in module bayescmd.abc.distances)}

\begin{fulllineitems}
\phantomsection\label{\detokenize{abc:bayescmd.abc.distances.euclidean_dist}}\pysiglinewithargsret{\sphinxcode{bayescmd.abc.distances.}\sphinxbfcode{euclidean\_dist}}{\emph{data1}, \emph{data2}}{}
Get the euclidean distance between two numpy arrays.
\begin{quote}\begin{description}
\item[{Parameters}] \leavevmode\begin{itemize}
\item {} 
\sphinxstyleliteralstrong{data1} (\sphinxstyleliteralemphasis{np.ndarray}) \textendash{} 
First data array.

The shape should match that of data2 and the number of rows should
match the number of model outputs i.e. 2 model outputs will be two
rows.


\item {} 
\sphinxstyleliteralstrong{data2} (\sphinxstyleliteralemphasis{np.ndarray}) \textendash{} 
Second data array.

The shape should match that of data1 and the number of rows should
match the number of model outputs i.e. 2 model outputs will be two
rows.


\end{itemize}

\item[{Returns}] \leavevmode
\sphinxstylestrong{d} \textendash{} Euclidean distance measure

\item[{Return type}] \leavevmode
float

\end{description}\end{quote}

\end{fulllineitems}

\index{get\_distance() (in module bayescmd.abc.distances)}

\begin{fulllineitems}
\phantomsection\label{\detokenize{abc:bayescmd.abc.distances.get_distance}}\pysiglinewithargsret{\sphinxcode{bayescmd.abc.distances.}\sphinxbfcode{get\_distance}}{\emph{actual\_data}, \emph{sim\_data}, \emph{targets}, \emph{zero\_flag}, \emph{distance='euclidean'}, \emph{normalise=False}}{}
Obtain  distance between two sets of data.

Get a distance as defined by \sphinxtitleref{distance} between two sets of data as well
as between each signal in the data.
\begin{quote}\begin{description}
\item[{Parameters}] \leavevmode\begin{itemize}
\item {} 
\sphinxstyleliteralstrong{actual\_data} (\sphinxstyleliteralemphasis{dict}) \textendash{} Dictionary of actual data, as generated by
\sphinxcode{bayescmd.abc.data\_import.import\_actual\_data()}

\item {} 
\sphinxstyleliteralstrong{sim\_data} (\sphinxstyleliteralemphasis{dict}) \textendash{} Dictionary of simulated data, as created by
{\hyperref[\detokenize{bcmdModel:bayescmd.bcmdModel.ModelBCMD.output_parse}]{\sphinxcrossref{\sphinxcode{bayescmd.bcmdModel.ModelBCMD.output\_parse()}}}}

\item {} 
\sphinxstyleliteralstrong{targets} (list of \sphinxcode{str}) \textendash{} List of model targets, which should all be strings.

\item {} 
\sphinxstyleliteralstrong{zero\_flag} (\sphinxstyleliteralemphasis{dict}) \textendash{} 
Dictionary of form target(\sphinxcode{str}): bool, where bool indicates
whether to zero that target.

Note: zero\_flag keys should match targets list.


\item {} 
\sphinxstyleliteralstrong{distance} (\sphinxstyleliteralemphasis{str}\sphinxstyleliteralemphasis{, }\sphinxstyleliteralemphasis{optional}) \textendash{} Name of distance measure to use. One of {[}‘euclidean’, ‘manhattan’,
‘MAE’, ‘MSE’{]}, where default is ‘euclidean’.

\item {} 
\sphinxstyleliteralstrong{normalise} (\sphinxstyleliteralemphasis{bool}\sphinxstyleliteralemphasis{, }\sphinxstyleliteralemphasis{optional}) \textendash{} Boolean flag to indicate whether the signals need normalising, default
is False. Current normalisation is done using z-score but that is
likely to change with time.

\end{itemize}

\item[{Returns}] \leavevmode

\sphinxstylestrong{distances} \textendash{}
\begin{description}
\item[{Dictionary of form:}] \leavevmode
\{‘TOTAL’: summed distance of all signals,
‘target1: distance of 1st target’,
…
‘targetN’: distance of Nth target
\}

\end{description}


\item[{Return type}] \leavevmode
dict

\end{description}\end{quote}

\end{fulllineitems}

\index{manhattan\_dist() (in module bayescmd.abc.distances)}

\begin{fulllineitems}
\phantomsection\label{\detokenize{abc:bayescmd.abc.distances.manhattan_dist}}\pysiglinewithargsret{\sphinxcode{bayescmd.abc.distances.}\sphinxbfcode{manhattan\_dist}}{\emph{data1}, \emph{data2}}{}
Get the Manhattan distance between two numpy arrays.
\begin{quote}\begin{description}
\item[{Parameters}] \leavevmode\begin{itemize}
\item {} 
\sphinxstyleliteralstrong{data1} (\sphinxstyleliteralemphasis{np.ndarray}) \textendash{} 
First data array.

The shape should match that of data2 and the number of rows should
match the number of model outputs i.e. 2 model outputs will be two
rows.


\item {} 
\sphinxstyleliteralstrong{data2} (\sphinxstyleliteralemphasis{np.ndarray}) \textendash{} 
Second data array.

The shape should match that of data1 and the number of rows should
match the number of model outputs i.e. 2 model outputs will be two
rows.


\end{itemize}

\item[{Returns}] \leavevmode
\sphinxstylestrong{d} \textendash{} Manhattan distance measure

\item[{Return type}] \leavevmode
float

\end{description}\end{quote}

\end{fulllineitems}

\index{mean\_absolute\_error\_dist() (in module bayescmd.abc.distances)}

\begin{fulllineitems}
\phantomsection\label{\detokenize{abc:bayescmd.abc.distances.mean_absolute_error_dist}}\pysiglinewithargsret{\sphinxcode{bayescmd.abc.distances.}\sphinxbfcode{mean\_absolute\_error\_dist}}{\emph{data1}, \emph{data2}}{}
Get the normalised manhattan distance between two numpy arrays.
\begin{quote}\begin{description}
\item[{Parameters}] \leavevmode\begin{itemize}
\item {} 
\sphinxstyleliteralstrong{data1} (\sphinxstyleliteralemphasis{np.ndarray}) \textendash{} 
First data array.

The shape should match that of data2 and the number of rows should
match the number of model outputs i.e. 2 model outputs will be two
rows.


\item {} 
\sphinxstyleliteralstrong{data2} (\sphinxstyleliteralemphasis{np.ndarray}) \textendash{} 
Second data array.

The shape should match that of data1 and the number of rows should
match the number of model outputs i.e. 2 model outputs will be two
rows.


\end{itemize}

\item[{Returns}] \leavevmode
\sphinxstylestrong{d} \textendash{} Normalised Manhattan distance measure

\item[{Return type}] \leavevmode
float

\end{description}\end{quote}

\end{fulllineitems}

\index{mean\_square\_error\_dist() (in module bayescmd.abc.distances)}

\begin{fulllineitems}
\phantomsection\label{\detokenize{abc:bayescmd.abc.distances.mean_square_error_dist}}\pysiglinewithargsret{\sphinxcode{bayescmd.abc.distances.}\sphinxbfcode{mean\_square\_error\_dist}}{\emph{data1}, \emph{data2}}{}
Get the Mean Square Error distance between two numpy arrays.
\begin{quote}\begin{description}
\item[{Parameters}] \leavevmode\begin{itemize}
\item {} 
\sphinxstyleliteralstrong{data1} (\sphinxstyleliteralemphasis{np.ndarray}) \textendash{} 
First data array.

The shape should match that of data2 and the number of rows should
match the number of model outputs i.e. 2 model outputs will be two
rows.


\item {} 
\sphinxstyleliteralstrong{data2} (\sphinxstyleliteralemphasis{np.ndarray}) \textendash{} 
Second data array.

The shape should match that of data1 and the number of rows should
match the number of model outputs i.e. 2 model outputs will be two
rows.


\end{itemize}

\item[{Returns}] \leavevmode
\sphinxstylestrong{d} \textendash{} Mean Square Error distance measure

\item[{Return type}] \leavevmode
float

\end{description}\end{quote}

\end{fulllineitems}

\index{zero\_array() (in module bayescmd.abc.distances)}

\begin{fulllineitems}
\phantomsection\label{\detokenize{abc:bayescmd.abc.distances.zero_array}}\pysiglinewithargsret{\sphinxcode{bayescmd.abc.distances.}\sphinxbfcode{zero\_array}}{\emph{array}, \emph{zero\_flag}}{}
Zero an array of data with its initial values.
\begin{quote}\begin{description}
\item[{Parameters}] \leavevmode\begin{itemize}
\item {} 
\sphinxstyleliteralstrong{array} (\sphinxstyleliteralemphasis{list}) \textendash{} List of data

\item {} 
\sphinxstyleliteralstrong{zero\_flags} (\sphinxstyleliteralemphasis{bool}) \textendash{} Boolean indicating if data needs zeroing

\end{itemize}

\item[{Returns}] \leavevmode
\sphinxstylestrong{zerod} \textendash{} Zero’d list

\item[{Return type}] \leavevmode
list

\end{description}\end{quote}

\end{fulllineitems}



\chapter{jsonParsing}
\label{\detokenize{jsonParsing:jsonparsing}}\label{\detokenize{jsonParsing::doc}}
..automodule:: bayescmd.jsonParsing.modelJSON


\chapter{Miscellaneous}
\label{\detokenize{misc:miscellaneous}}\label{\detokenize{misc::doc}}
Here you will find a number of useful functions that are used throughout
the general BayesCMD package.


\chapter{Indices and tables}
\label{\detokenize{index:indices-and-tables}}\begin{itemize}
\item {} 
\DUrole{xref,std,std-ref}{genindex}

\item {} 
\DUrole{xref,std,std-ref}{modindex}

\item {} 
\DUrole{xref,std,std-ref}{search}

\end{itemize}


\renewcommand{\indexname}{Python Module Index}
\begin{sphinxtheindex}
\def\bigletter#1{{\Large\sffamily#1}\nopagebreak\vspace{1mm}}
\bigletter{b}
\item {\sphinxstyleindexentry{bayescmd.abc}}\sphinxstyleindexpageref{abc:\detokenize{module-bayescmd.abc}}
\item {\sphinxstyleindexentry{bayescmd.abc.distances}}\sphinxstyleindexpageref{abc:\detokenize{module-bayescmd.abc.distances}}
\item {\sphinxstyleindexentry{bayescmd.bcmdModel.bcmd\_model}}\sphinxstyleindexpageref{bcmdModel:\detokenize{module-bayescmd.bcmdModel.bcmd_model}}
\item {\sphinxstyleindexentry{bayescmd.bcmdModel.input\_creation}}\sphinxstyleindexpageref{bcmdModel:\detokenize{module-bayescmd.bcmdModel.input_creation}}
\end{sphinxtheindex}

\renewcommand{\indexname}{Index}
\printindex
\end{document}